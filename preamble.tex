% ---------------- Dempe unødvendige advarsler  ---------------------------------------
\WarningFilter{latex}{Command \showhyphens has changed}

% ----------------------- Formatering ---------------------------------------
% --- Avsnitt ---
\setlength{\parindent}{0pt}                                                               % Ingen innrykk på første linje i avsnitt
\setlength{\parskip}{0.6\baselineskip plus 2pt minus 1pt}                                 % Dynamisk luft mellom avsnitt (kan strekkes/krympes)

% --- Linjeavstand (litt løsere) --- 
\setstretch{1.10}                                                                         % Litt løsere linjeavstand (typisk 1.10–1.15)

% --- Luft rundt display-matte (equation/align/...) ---
\makeatletter                                                                             
\g@addto@macro\normalsize{%                                                               % Justér standard-verdier for normalsize
  \setlength\abovedisplayskip{-0.3\baselineskip plus 1pt minus 1pt}                        % Luft over display-likninger
  \setlength\belowdisplayskip{1.3\baselineskip plus 1pt minus 1pt}                        % Luft under display-likninger
  \setlength\abovedisplayshortskip{-0.3\baselineskip plus 1pt minus 1pt}                   % Når linja er “kort” før display
  \setlength\belowdisplayshortskip{1.3\baselineskip plus 1pt minus 1pt}                   % Tilsvarende under
}
\makeatother                                                                             

% --- Seksjonsavstand (med titlesec) ---
\titlespacing*{\section}
  {0pt}{2\baselineskip plus .4\baselineskip minus .2\baselineskip}{0.2\baselineskip}    % Venstreinnrykk, før- og etter-avstand
\titlespacing*{\subsection}
  {0pt}{1.1\baselineskip plus .3\baselineskip minus .2\baselineskip}{0\baselineskip}    % Litt strammere for subsection

% --- Lister (med enumitem) ---
\setlist{itemsep=0.4\baselineskip plus 2pt minus 1pt, topsep=0.5\baselineskip}   

% --- Innholdsfortegnelse (med tocloft) ---
\setlength{\cftbeforesecskip}{0.25\baselineskip}                                          % Luft før hver section-linje i ToC
\setlength{\cftbeforesubsecskip}{-0.15\baselineskip}                                      % Luft før hver subsection-linje i ToC

%------------------------Pakke-konflikt-----------------------------
% siunitx vs physics: bruk siunitx sin \qty
\AtBeginDocument{\RenewCommandCopy\qty\SI}
\ExplSyntaxOn
\msg_redirect_name:nnn { siunitx } { physics-pkg } { none }
\ExplSyntaxOff

\emergencystretch=1em

\pgfplotsset{compat=1.18}

                   
% -------------- Stil ------------------------------------------------
\hypersetup{
  colorlinks=true,
  linkcolor=black,
  urlcolor=black,
  citecolor=black,
  pdfauthor={\authornames},
  pdftitle={\documenttitle}
}

\addto\captionsnorsk{%
  \renewcommand{\contentsname}{Innholdsfortegnelse}%      Endrer "Innhold" til "Innholdsfortegnelse"
}
% Fargekoding for kode i tekst med \listinline og begin{lstlisting}
\definecolor{vscodeKeyword}{HTML}{569CD6}  % blå keywords
\definecolor{vscodeString}{HTML}{CE9178}   % salmon strenger  ← denne er nøkkelen
\definecolor{vscodeComment}{HTML}{6A9955}  % grønn kommentar
\definecolor{vscodeType}{HTML}{4EC9B0}     % teal typer/builtins
\definecolor{vscodeText}{HTML}{202020}     % vanlig tekst

\lstset{
  language=Python,
  backgroundcolor=\color{black!5},
  basicstyle=\ttfamily\small\color{vscodeText},
  keywordstyle=\color{vscodeKeyword}\bfseries,
  stringstyle=\color{vscodeString},         % <-- mer “riktig” string-farge
  commentstyle=\color{vscodeComment},
  emph={self,np,plt,pd,range,len},
  emphstyle=\color{vscodeType},
  showstringspaces=false,
  breaklines=true,
  tabsize=4,
  frame=none,
  numbers=none,
  aboveskip=0.9\baselineskip,
  belowskip=1.1\baselineskip,
}


\titleformat{\section}{\large\bfseries}{\thesection.}{0.5em}{}
\titleformat{\subsection}{\normalsize\bfseries}{\thesubsection}{0.5em}{}
\titleformat{\subsubsection}{\normalsize\itshape}{\thesubsubsection}{0.5em}{}