\subsection{Tolkning av resultater}
\subsubsection{Kort oppsummering av funn}
I dette arbeidet er bølgelikningen for en streng utledet både teoretisk og implementert numerisk. Utgangspunktet er en fysisk modell av en streng festet i begge ender, og påvirket av en spenning, der man ved hjelp av Newtons lover og Taylor-utvikling kommer fram til bølgelikningen:

\begin{equation*}
\frac{\partial^2 u}{\partial t^2} = \frac{T}{\rho} \frac{\partial^2 u}{\partial x^2}
\end{equation*}
Her er \( T \) spenningen i strengen, \( \rho \) massetettheten, og \( u(x, t) \) beskriver utslaget til strengen. Bølgehastigheten defineres som

\begin{equation*}
c^2 = \frac{T}{\rho}
\end{equation*}
Den teoretiske løsningen gjøres ved separasjon av variable, hvor løsningen antas å være et produkt av en romlig funksjon \( F(x) \) og en tidsavhengig funksjon \( G(t) \). Dette fører til to separate differensiallikninger, som løses med sinus- og cosinus-funksjoner. Ved å bruke grensebetingelser for en streng som er fast i begge ender, finner man en løsning som uttrykkes som en uendelig sum (Fourier-rekke) av normalmoduser:

\begin{equation*}
u(x, t) = \sum_{n=1}^{\infty} c_n \sin\left(\frac{n\pi x}{l}\right)\cos\left(\frac{n\pi c t}{l}\right)
\end{equation*}
Den numeriske løsningen er implementert i Python, hvor bibliotekene \texttt{NumPy} og \texttt{Matplotlib} benyttes til henholdsvis beregninger og visualisering. Startformen til strengen ved \( t = 0 \) defineres som en trekantform (\textit{pluck}), som er en forenklet, men effektiv modell av hvordan en streng kan plukkes i praksis. Videre brukes \texttt{Matplotlib.animation} for å vise hvordan strengen beveger seg over tid.

Til slutt beregnes de teoretiske egenfrekvensene for strengen. Når strengen er festet i begge ender, kan den kun vibrere i bestemte mønstre, kalt normalmoduser. Grunnmodusen har bølgelengde

\begin{equation*}
\lambda_1 = 2L
\end{equation*}
og tilhørende frekvens er gitt av Mersenne's formel:

\begin{equation*}
f_1 = \frac{1}{2L} \sqrt{\frac{T}{\mu}}
\end{equation*}
Høyere moduser har frekvenser som er heltallsmultipler av grunnmodusen:

\begin{equation*}
f_n = n f_1 = \frac{n}{2L} \sqrt{\frac{T}{\mu}}, \quad n = 1, 2, 3, \dots
\end{equation*}
\subsubsection{Styrker og svakheter i modellen}
Modellen som er brukt i dette arbeidet har flere tydelige styrker, men også noen svakheter. Det er 
oppgaven til dette kapittelet å forstå de ulike styrkene og svakhetene, og hvordan disse påvirker 
resultatet.

En av modellens største styrker er at den bygger på de fundamentale prinsippene i bølgelikningen for
et ideelt system. Et ideelt system innebærer at man ser bort fra faktorer som demping, stivhet i strengen
og ikke-lineære effekter. Dette gjør det mulig å fokusere på de grunnleggende egenskapene ved 
bølgebevegelsen, og gir en klar forståelse av hvordan strengen oppfører seg under ideelle forhold. 
Den numeriske implementasjonen er også en styrke, da den gir en visuell representasjon av 
bølgebevegelsen som kan være lettere å forstå enn rene matematiske uttrykk.

Samtidig har modellen enkelte svakheter som må tas i betraktning. For det første tar den ikke hensyn
til at energien ikke er bevart i et reelt system. I praksis vil en streng miste energi over tid på grunn av
demping og friksjon, noe som ikke er inkludert i denne modellen. Dette fører til at de numeriske resultatene kan avvike
fra virkelige observasjoner over lengre tid. En annen svakhet er at modellen antar en perfekt trekantformet startbetingelse,
som ikke nødvendigvis representerer hvordan en streng faktisk plukkes i praksis. Dette kan påvirke de initielle 
vibrasjonsmønstrene og dermed resultatene. Det er også verdt å merke seg at modellen forutsetter at strengen er homogen 
og fullstendig elastisk, noe som ikke alltid gjelder for virkelige materialer. Variasjoner i tetthet eller spenning kan føre til komplekse 
bølgefenomener som ikke fanges opp av denne forenklede modellen.

Et annet forhold er at den numeriske metoden kun benytter 60 moduser ($N=60$) i Fourier-rekken for å tilnærme 
løsningen. Selv om dette er tilstrekkelig for å fange opp hovedtrekkene i bølgebevegelsen, kan det 
føre til at finere detaljer og høyere frekvenskomponenter ikke blir representert nøyaktig. Dette kan være spesielt merkbart
dersom strengen plukkes på en måte som introduserer høyfrekvente vibrasjoner.

Samlet sett gir modellen en god balanse mellom kompleksitet og forståelse. Den er godt egnet til å illustrere
hvordan spenning, massetetthet og lengde påvirker bølgebevegelsen, men det er viktig å være bevisst på dens
begrensninger. For mer realistiske simuleringer av fysiske systemer vil det være nødvendig å inkludere faktorer
som demping, materialvariasjoner og mer realistiske startbetingelser.

\subsubsection{Effekt av parametervariasjon}

Bølgebevegelsen til en streng er sterkt avhengig av de fysiske parameterne som inngår i bølgelikningen. Spenningen \( T \), massetettheten per lengdeenhet \( \mu \), og strengens lengde \( L \) påvirker både bølgehastigheten og egenfrekvensene.

Bølgehastigheten \( c \) i strengen bestemmes av forholdet mellom spenningen og massetettheten, og uttrykkes som

\begin{equation*}
c = \sqrt{\frac{T}{\mu}}.
\end{equation*}
Dette betyr at en økning i spenningen \( T \) fører til at bølgene beveger seg raskere langs strengen, mens en økning i massetettheten \( \mu \) gir en tregere bølgebevegelse. 

Når det gjelder strengens vibrasjonsmønstre, kan den kun svinge ved bestemte frekvenser, kalt egenfrekvenser, på grunn av de faste endebetingelsene. Grunnfrekvensen \( f_1 \) er gitt ved Mersennes formel

\begin{equation*}
f_1 = \frac{1}{2L} \sqrt{\frac{T}{\mu}},
\end{equation*}
og de høyere normalmodusene følger som heltallsmultipler av denne:

\begin{equation*}
f_n = \frac{n}{2L} \sqrt{\frac{T}{\mu}}, \quad n=1,2,3,\ldots
\end{equation*}
\clearpage
Fra disse uttrykkene kan man se at økning i spenning \( T \) fører til høyere frekvenser, noe som tilsvarer en lysere tone når strengen vibrerer. Økt massetetthet \( \mu \) gir lavere frekvenser og dermed en dypere tone. Lengden på strengen \( L \) er også avgjørende. En lengre streng har lavere frekvenser, fordi bølgelengdene øker og dermed tonehøyden synker.

I den numeriske simuleringen reflekteres disse effektene tydelig. Ved å øke \( T \) ser man at strengen svinger raskere i animasjonen, mens økt \( \mu \) eller \( L \) resulterer i tregere bevegelser og lavere frekvenser.