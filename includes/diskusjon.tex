\subsection{Tolkning av resultater}
\subsubsection{Kort oppsummering av funn}
I dette arbeidet er bølgelikningen for en streng utledet både teoretisk og implementert numerisk. Utgangspunktet er en fysisk modell av en streng festet i begge ender, og påvirket av en spenning, der man ved hjelp av Newtons lover og Taylor-utvikling kommer fram til bølgelikningen:

\begin{equation*}
\frac{\partial^2 u}{\partial t^2} = \frac{T}{\rho} \frac{\partial^2 u}{\partial x^2}
\end{equation*}

Her er \( T \) spenningen i strengen, \( \rho \) massetettheten, og \( u(x, t) \) beskriver utslaget til strengen. Bølgehastigheten defineres som

\begin{equation*}
c^2 = \frac{T}{\rho}
\end{equation*}

Den teoretiske løsningen gjøres ved separasjon av variable, hvor løsningen antas å være et produkt av en romlig funksjon \( F(x) \) og en tidsavhengig funksjon \( G(t) \). Dette fører til to separate differensiallikninger, som løses med sinus- og cosinus-funksjoner. Ved å bruke grensebetingelser for en streng som er fast i begge ender, finner man en løsning som uttrykkes som en uendelig sum (Fourier-rekke) av normalmoduser:

\begin{equation*}
u(x, t) = \sum_{n=1}^{\infty} c_n \sin\left(\frac{n\pi x}{l}\right)\cos\left(\frac{n\pi c t}{l}\right)
\end{equation*}

Den numeriske løsningen er implementert i Python, hvor bibliotekene \texttt{NumPy} og \texttt{Matplotlib} benyttes til henholdsvis beregninger og visualisering. Startformen til strengen ved \( t = 0 \) defineres som en trekantform (\textit{pluck}), som er en forenklet, men effektiv modell av hvordan en streng kan plukkes i praksis. Videre brukes \texttt{Matplotlib.animation} for å vise hvordan strengen beveger seg over tid.

Til slutt beregnes de teoretiske egenfrekvensene for strengen. Når strengen er festet i begge ender, kan den kun vibrere i bestemte mønstre, kalt normalmoduser. Grunnmodusen har bølgelengde

\begin{equation*}
\lambda_1 = 2L
\end{equation*}

og tilhørende frekvens er gitt av Mersenne's formel:

\begin{equation*}
f_1 = \frac{1}{2L} \sqrt{\frac{T}{\mu}}
\end{equation*}

Høyere moduser har frekvenser som er heltallsmultipler av grunnmodusen:

\begin{equation*}
f_n = n f_1 = \frac{n}{2L} \sqrt{\frac{T}{\mu}}, \quad n = 1, 2, 3, \dots
\end{equation*}




\subsubsection{Styrker og svakheter i modellen}
\dots



\subsubsection{Effekt av parametervariasjon}

Bølgebevegelsen til en streng er sterkt avhengig av de fysiske parameterne som inngår i bølgelikningen. Spenningen \( T \), massetettheten per lengdeenhet \( \mu \), og strengens lengde \( L \) påvirker både bølgehastigheten og egenfrekvensene.

Bølgehastigheten \( c \) i strengen bestemmes av forholdet mellom spenningen og massetettheten, og uttrykkes som

\begin{equation*}
c = \sqrt{\frac{T}{\mu}}.
\end{equation*}
Dette betyr at en økning i spenningen \( T \) fører til at bølgene beveger seg raskere langs strengen, mens en økning i massetettheten \( \mu \) gir en tregere bølgebevegelse. 

Når det gjelder strengens vibrasjonsmønstre, kan den kun svinge ved bestemte frekvenser, kalt egenfrekvenser, på grunn av de faste endebetingelsene. Grunnfrekvensen \( f_1 \) er gitt ved Mersennes formel

\begin{equation*}
f_1 = \frac{1}{2L} \sqrt{\frac{T}{\mu}},
\end{equation*}
og de høyere normalmodusene følger som heltallsmultipler av denne:

\begin{equation*}
f_n = \frac{n}{2L} \sqrt{\frac{T}{\mu}}, \quad n=1,2,3,\ldots
\end{equation*}

Fra disse uttrykkene kan man se at økning i spenning \( T \) fører til høyere frekvenser, noe som tilsvarer en lysere tone når strengen vibrerer. Økt massetetthet \( \mu \) gir lavere frekvenser og dermed en dypere tone. Lengden på strengen \( L \) er også avgjørende. En lengre streng har lavere frekvenser, fordi bølgelengdene øker og dermed tonehøyden synker.

I den numeriske simuleringen reflekteres disse effektene tydelig. Ved å øke \( T \) ser man at strengen svinger raskere i animasjonen, mens økt \( \mu \) eller \( L \) resulterer i tregere bevegelser og lavere frekvenser.