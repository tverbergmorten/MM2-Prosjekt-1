\subsection{Teoretiske resultater}
\subsubsection{Uteldning av egenfrekvenser}
For en streng som er festet i to punkter er hastigheten $v$ gitt ved:

\begin{equation*}
    v = \sqrt{\frac{T}{\mu}}
\end{equation*}
Hvor $T$ er spenningen i strengen og $\mu$ er massetettheten. 

Forholdet mellom hastighet ($v$), frekvens ($f$) og bølgelengde ($\lambda$) er gitt ved

\begin{equation*}
    v = f \lambda
\end{equation*}  

Når strengen er festet i begge ender, kan den bare svinge i bestemte mønstre, kalt normalmoduser.
Den enkleste normalmodusen er grunnmodusen, som har en bølgelengde som er dobbelt så lang som strengen:

\begin{equation*}
    \lambda_1 = 2L
\end{equation*}
Hvor $L$ er lengden på strengen.

Frekvensen til grunnmodusen kan dermed uttrykkes som:

\begin{equation*}
    f_1 = \frac{v}{\lambda} = \frac{1}{2L} \sqrt{\frac{T}{\mu}}
\end{equation*}
Dette er Mersenne's likning.




\subsubsection{Normalmoduser}

Det er uendelig mange normalmoduser for en streng som er festet i begge ender.
Som nevnt tidligere er grunnmodusen den enkleste normalmodusen og er den med lavest frekvens.

Hver påfølgende normalmodus har en frekvens som er et helt multiplum av grunnmodusen,
hvor $n$ er et positivt heltall som representerer modustallet.

Dette kan utledes ved å merke seg at for den $n$-te normalmodusen, er bølgelengden gitt ved:

\begin{equation*}
    \lambda_n = \frac{2L}{n}
\end{equation*}
Ved å sette dette inn i bølgeformelen $v = f \lambda$, får vi:

\begin{equation*}
    f_n = \frac{v}{\lambda_n} = \frac{1}{\frac{2L}{n}} \sqrt{\frac{T}{\mu}} = \frac{n}{2L} \sqrt{\frac{T}{\mu}}
\end{equation*}
Dermed er frekvensen til den $n$-te normalmodusen:

\begin{equation*}
    f_n = n f_1 = \frac{n}{2L} \sqrt{\frac{T}{\mu}}
\end{equation*}
Hvor $n = 1, 2, 3, \ldots$.




\subsection{Numeriske resultater}
\subsubsection{Visualisering av strengens bevegelse}
\begin{figure}[h!]
\centering
\begin{tikzpicture}[scale=1.0]
  % aksene
  \draw[->] (-0.5,0) -- (11,0) node[right] {$x$};
  \draw[->] (0,-2.5) -- (0,2.5) node[above] {$y$};

  % parametre
  \def\A{2} % amplitude
  \def\nodes{0, 5, 10} % noder

  % stående bølge: y = 2*sin(x), snapshot
  \draw[blue, thick, samples=200, domain=0:10] 
        plot (\x, {\A*sin(deg(pi*\x/5))});

  % markere noder
  \foreach \x in \nodes {
    \fill[black] (\x,0) circle (2pt);
  }

  % litt forskjøvet bølge (stiplet)
  \draw[red, thick, dashed, samples=200, domain=0:10] 
        plot (\x, {\A*sin(deg(pi*\x/5))*0.7});
\end{tikzpicture}
\end{figure}




\subsubsection{Sammenligning med teoretiske resultater}
\dots

\subsubsection{Parametervariasjon}
\dots