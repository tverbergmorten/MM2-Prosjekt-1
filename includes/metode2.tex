\subsection{Beskrivelse av kode for numerisk løsning}
\subsubsection{Pakker, variabler og parametere}
Først importeres nødvendige pakker for å kunne utføre numeriske beregninger og plotte grafiske fremstillinger. Her
tar vi i bruk numpy for numeriske operasjoner, matplotlib.pyplot for plotting, og matplotlib.animation for å lage 
animasjoner slik som forklart i forrige seksjon. Deretter setter vi konfigurajsonsparametere som definerer
fysiske egenskaper til strengen, samt parametere for den numeriske løsningen. Dette inkluderer lengden $L$ på strengen,
bølgehastigheten $c$ og tiden $s$ vi ønsker å simulere. Videre defineres antall punkter $N$ som skal brukes til å dele 
opp strengen. Dette påvirker nøyaktigheten til den numeriske løsningen, der flere punkter gir en mer nøyaktig løsning,
men øker beregningstid og ressursbruk. Videre definerer vi modi for funksjonen, som er antall bølgetopper som skal være 
tilstede for å lage startformen, hvor en høyere verdi gir skarpere kanter og høyere detaljnivå. \parencite{bølgeSimulering}

Videre setter vi opp betingelser for den grafiske fremstillingen, slik som FPS (frames per second) for animasjonen, og
tykkelsen for linjen på grafen. For å sette startformen på grafen setter vi 

\begin{verbatim}
  initial_shape_type = "pluck"
  pluck_width: float = 0.15
\end{verbatim}

hvor \verb|"pluck"| indikerer at grafen (i dette tilfellet en streng) skal strekkes, og \verb|pluck_width| definerer hvor bredt området som strekkes er.

\subsubsection{Initialbetingelser funksjonen}
Hensikten med denne funksjonen er å definere startformen til strengen ved $t=0$, som vil si at funksjonen returnerer $u(x,0)$.
Funksjonen tar inn en parameter \verb|x|, som er en numpy array med posisjoner langs strengen. Avhengig av verdien til
\verb|initial_shape_type|, vil funksjonen returnere forskjellige startformer. I dette tilfellet er det \verb|"pluck"| 
som er implementert som vil gi funksjonen en trekant-/teltformet startform, hvor plukkposisjonen bestemmes av:

\begin{verbatim}
  elif initial_shape_type == "pluck":
    center = pluck_position * length_m
    width = pluck_width * length_m
    return np.clip(1.0 - np.abs(x - center) / width, 0.0, 1.0)
\end{verbatim}

hvis du setter inn for verdi i vi har satt i starten av koden, vil dette si at strengen blir plukket slik som 
illustrert i figuren nedenfor:

\begin{figure}[h]
    \centering
    \begin{tikzpicture}
        \begin{axis}[
        width=0.9\textwidth,
        height=0.5\textwidth,
        axis lines=left,
        xlabel={$x$ [m]},
        ylabel={$u(x,0)$ [m]},
        xmin=0, xmax=1.0,
        ymin=-1.1, ymax=1.1,
        xtick=1,
        ytick={-1, 0, 1}
        ]
        

        \draw (0,0) -- (0.3,0) -- (0.4,1) -- (0.5,0) -- (1,0);
        \fill (0.4,1) circle (2pt);


        \end{axis}
    \end{tikzpicture}
    \caption{Startform av strengen ved $t=0$ når den blir plukket ved $x=0.3$ m.}
\end{figure}



\subsubsection{}
\dots
