\subsection{Fysiske størrelser}

\subsubsection{\texorpdfstring{$u(x,t)$}{u(x,t)}}
$u(x,t)$ beskriver utslaget eller forskyvningen til bølgen i et punkt $x$ til et tidspunkt $t$. 
For en streng representerer dette hvor langt strengen beveger seg i vertikal retning fra likevektsposisjonen.

\subsubsection{Tid \texorpdfstring{$t$}{t}}
Tiden $t$ angir øyeblikket vi observerer bølgen på. 
Den er en kontinuerlig variabel som bestemmer hvordan bølgen utvikler seg.

\subsubsection{Romkoordinatet \texorpdfstring{$x$}{x}}
Koordinatet $x$ beskriver posisjonen langs mediet der bølgen brer seg, for eksempel langs en streng. 
Det gjør at vi kan bestemme bølgens form på forskjellige steder til et gitt tidspunkt.

\subsubsection{Bølgefarten \texorpdfstring{$c$}{c}}
Bølgefarten $c$ forteller hvor raskt bølgeforstyrrelsen brer seg gjennom mediet. 
Høyere bølgefart betyr at forstyrrelser forplanter seg raskere. 
Bølgefarten bestemmes av mediets egenskaper, blant annet strekkspenning og masse per lengdeenhet.

\subsubsection{Strekkraft \texorpdfstring{$T$}{T}}
Strekkraften $T$ beskriver kraften som strengen er strammet med. 
Jo større strekkraft, desto høyere blir bølgefarten i strengen, siden den leder vibrasjoner mer effektivt.
På engelsk omtales dette som \emph{tension}, og defineres som en kraft langs lengden av et fleksibelt medium, 
slik som en streng. \parencite{libretextsTension}

\subsubsection{Masse per lengde \texorpdfstring{$\mu$}{mu}}
$\mu$ er massen fordelt per lengdeenhet av strengen. 
En større $\mu$ betyr at strengen er tyngre, og bølgefarten blir dermed lavere.

\subsection{Utledning av bølgeligningen}
\subsubsection{Krefter på en strengbit, newtons 2.lov}
\dots

\subsubsection{Trinnvis gjennomgang av utledningen}
\dots
