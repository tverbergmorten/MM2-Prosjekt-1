\subsection{Fysiske størrelser}

\subsubsection{\texorpdfstring{$u(x,t)$}{u(x,t)}}
$u(x,t)$ beskriver forskyvningen til et punkt på strengen ved posisjon $x$ og tidspunkt $t$. 
For en streng representerer dette hvor langt punktet beveger seg i vertikal retning fra sin likevektsposisjon. 
I utledningen av bølgeligningen fra Newtons 2.\ lov er $u(x,t)$ den størrelsen som beskriver bevegelsen til et lite strengsegment som påvirkes av strekkreftene i hver ende \parencite{LamarVibratingString}.

\subsubsection{Tid \texorpdfstring{$t$}{t}}
Tiden $t$ angir øyeblikket vi observerer bølgen på. 
Den er en kontinuerlig variabel som beskriver hvordan bevegelsen utvikler seg over tid. 
Når vi løser bølgeligningen, viser variasjonen i $t$ hvordan energi og forskyvning forplanter seg gjennom systemet \parencite{LamarWaveEquationSection}.

\subsubsection{Romkoordinatet \texorpdfstring{$x$}{x}}
Koordinatet $x$ beskriver posisjonen langs strengen der bølgen brer seg. 
Ved å holde $t$ konstant kan vi se hvordan utslaget varierer i rommet. 
Dette gjør det mulig å beskrive bølgens form og fase langs strengen på et gitt tidspunkt \parencite{UBCWaveSeparation}.

\subsubsection{Bølgefarten \texorpdfstring{$c$}{c}}
Bølgefarten $c$ forteller hvor raskt en forstyrrelse brer seg langs strengen. 
Ut fra Newtons 2.\ lov for et lite strengsegment kan bølgefarten c uttrykkes som (\ref{eq:bølgefart}).

der $T$ er strekkraften i strengen og $\rho$ er massen per lengdeenhet. 
En streng med høy spenning eller lav masse per lengde vil derfor ha høyere bølgefart. 
Denne sammenhengen gjelder generelt for mekaniske bølger i strenger og kabler \parencite{LibreTextsWaveSpeed}.

\subsubsection{Strekkraft \texorpdfstring{$T$}{T}}
Strekkraften $T$ beskriver hvor stramt strengen er spent. 
Den virker langs strengens lengde og sørger for at bølgebevegelsen kan forplante seg. 
Økt strekkraft gir økt bølgefart, fordi systemet reagerer raskere på små forskyvninger. 
Dette er grunnen til at en gitarstreng med høyere spenning gir en høyere tone, siden bølgefarten og dermed frekvensen øker. 
\parencite{NTNUBolgelikning}

\subsubsection{Masse per volum \texorpdfstring{$\rho$}{rho}}
$\rho$ er massen fordelt per lengdeenhet av strengen, målt i $\text{kg/m³}$. 
En større $\rho$ betyr at strengen er tyngre og vanskeligere å sette i bevegelse, slik at bølgefarten blir lavere. 
Tynne og lette strenger vibrerer raskere og produserer derfor høyere toner enn tykke og tunge strenger \parencite{UCFStringLinearDensity}.

\subsection{Fourierrekker og halvperiodiske utvidelser}

En streng som settes i bevegelse, kan beskrives som en superposisjon av mange enkle svingninger.  
Hver svingning har sin egen frekvens og amplitude, og summen av disse danner den totale bevegelsen.  
Dette kan matematisk uttrykkes som en \textit{Fourierrekke}, som er et av de mest sentrale verktøyene for å analysere og løse bølgeligningen.  
\clearpage

Ideen bak Fourier-representasjonen er at selv et komplisert utslag på strengen, for eksempel et tilfeldig startbøyningsmønster, kan deles opp i en sum av sinus- og cosinusfunksjoner.  
På den måten kan man løse bølgeligningen for hver enkelt harmonisk komponent, og deretter summere løsningene for å finne den totale bevegelsen.  

Selv om Fourier-rekker vanligvis brukes for periodiske fenomener, kan de også brukes når funksjonen er definert på et begrenset intervall, for eksempel $x \in [0,L]$.  
Da utvider man funksjonen \textit{halvperiodisk}, slik at intervallet $[0,L]$ betraktes som en halvperiode av en større periodisk funksjon.  
Dette er spesielt nyttig for strenger som er festet i begge ender, siden utslaget der må være null.  
En slik funksjon er derfor \textit{odde} (antisymmetrisk), og beskrives med en ren sinusrekke. \parencite{intmathHalfRange} 

Fourierrekken for en slik odde, halvperiodisk funksjon $f(x)$ definert på intervallet $[0,L]$ blir da

\begin{equation*}
f(x) = \sum_{n=1}^{\infty} c_n \sin\left(\frac{n\pi x}{L}\right), \quad x \in [0,L]
\end{equation*}

der koeffisientene $c_n$ beregnes som

\begin{equation*}
c_n = \frac{2}{L} \int_{0}^{L} f(x) \sin\left(\frac{n\pi x}{L}\right) dx
\end{equation*}

Ved å bruke Fourierrekker på denne formen kan man uttrykke et vilkårlig startutslag $u(x,0)$ som en sum av sinusfunksjoner.  
Deretter kan bølgeligningen løses separat for hver sinuskomponent, og til slutt summeres alle bidragene for å finne den fullstendige løsningen.  
Denne metoden gir en effektiv og nøyaktig måte å beskrive bølgebevegelser på, og danner grunnlaget for den generelle løsningen:

\begin{equation*}
u(x,t) = \sum_{n=1}^{\infty} \left[a_n \cos(\omega_n t) + b_n \sin(\omega_n t)\right] \sin\left(\frac{n\pi x}{L}\right)
\end{equation*}

der $\omega_n = \frac{n\pi c}{L}$ er egenfrekvensene som bestemmes av bølgefarten $c$ og strengens lengde $L$.
\clearpage
\subsection{Egenfrekvens og egenmoduser}

Når en streng svinger, skjer dette kun ved bestemte frekvenser som tilfredsstiller randbetingelsene $u(0,t)=u(L,t)=0$.  
Disse kalles \textit{egenfrekvenser}, og de tilsvarende bølgeformene kalles \textit{egenmoduser}.  
Egenfrekvensene bestemmes av strengens lengde $L$, bølgefart $c$, og heltallet $n$:

\begin{equation*}
\omega_n = \frac{n\pi c}{L}, \quad n = 1,2,3,\dots
\end{equation*}

Den tilhørende romlige egenfunksjonen er

\begin{equation*}
X_n(x) = \sin\left(\frac{n\pi x}{L}\right)
\end{equation*}

Den første modusen ($n=1$) kalles \textit{grunnfrekvensen}, høyere $n$ gir overtoner eller harmoniske frekvenser.  
Disse danner grunnlaget for klangen i musikkinstrumenter og for Fourier-representasjonen av bølgeløsningen \parencite{physicsClassroomFundamental}

\subsection{Utledning av egenfrekvenser}
For en streng som er festet i to punkter er hastigheten $c$ gitt ved:

\begin{equation*}
    c = \sqrt{\frac{T}{\mu}}
\end{equation*}
Hvor $T$ er spenningen i strengen og $\mu$ er massetettheten, og forholdet mellom hastighet
($c$), frekvens ($f$) og bølgelengde ($\lambda$) er gitt ved

\begin{equation*}
    c = f \lambda
\end{equation*}  

Når strengen er festet i begge ender, kan den bare svinge i bestemte mønstre, kalt normalmoduser.
Den enkleste normalmodusen er grunnmodusen, der bølgelengden er det dobbelte av strengen:

\begin{equation*}
    \lambda_1 = 2L
\end{equation*}
Hvor $L$ er lengden på strengen.
Frekvensen til grunnmodusen kan dermed uttrykkes som:

\begin{equation*}
    f_1 = \frac{c}{\lambda} = \frac{1}{2L} \sqrt{\frac{T}{\mu}}
\end{equation*}
Dette er Mersenne's likning. \parencite{MersennesLaws}

\subsection{Stående og løpende bølger}

Løsningen på bølgeligningen kan representere både \textit{løpende} og \textit{stående} bølger.  
En løpende bølge har formen

\begin{equation*}
u(x,t) = f(x - ct) + g(x + ct)
\end{equation*}

der $f$ og $g$ beskriver bølger som beveger seg i motsatt retning med fart $c$.  
Når to slike bølger interfererer, kan de danne et mønster som ikke forplanter seg, men oscillerer lokalt — dette kalles en \textit{stående bølge}.  

For en streng med faste endepunkter oppstår stående bølger når bølgelengden tilfredsstiller betingelsen

\begin{equation*}
\lambda_n = \frac{2L}{n}, \quad n = 1, 2, 3, \dots
\end{equation*}

som betyr at strengen bare kan svinge i bestemte \textit{normalmoduser}.  
Disse gir opphav til de karakteristiske egenfrekvensene

\begin{equation*}
f_n = \frac{n c}{2L},
\end{equation*}

der $L$ er strengens lengde.  
Fenomenet med stående bølger forklarer hvorfor musikkinstrumenter som gitarer og fioliner kan produsere toner med bestemte frekvenser — det er et resultat av interferens mellom bølger som reflekteres ved strengens ender \parencite{libretextsStandingWaves}.

\subsection{Resonans}

Resonans oppstår når et system utsettes for en periodisk kraft med en frekvens som samsvarer med en av systemets egenfrekvenser.  
I tilfelle av en streng vil en slik resonans føre til at amplituden øker betydelig.  
Dette er grunnlaget for lydforsterkning i musikkinstrumenter, der strengens naturlige frekvenser forsterkes av klangkassen.  

Matematisk kan resonans betraktes som at en tvungen løsning av bølgeligningen får maksimal amplitude når den påtrykte frekvensen $\omega$ er lik $\omega_n$ for en egenmodus.  
Fenomenet er sentralt i fysikk og ingeniørfag fordi det kan føre til både ønskede og uønskede svingninger \parencite{libretextsStringResonance}.

I dette prosjektet har det ikke blitt inkludert resonans som en del av den matematiske modellen, ettersom strengens bevegelse studeres over et kort tidsintervall uten ytre periodisk pådriv.
Effekten av resonans blir først merkbar ved langvarig eller tvungen svingning, og har derfor lite utslag i simuleringene i denne rapporten.
Likevel er fenomenet tatt med her for å vise forståelse for hvordan resonans henger sammen med egenfrekvensene som er sentrale i bølgeløsningen.

\subsection{Demping}

I virkelige systemer vil energi gradvis tapes på grunn av indre friksjon og luftmotstand.  
Dette gjør at amplituden av bølgen reduseres over tid.  
En enkel modell for dempet bølgebevegelse kan uttrykkes som

\begin{equation*}
\frac{\partial^2 u}{\partial t^2} + 2\beta \frac{\partial u}{\partial t} = c^2 \frac{\partial^2 u}{\partial x^2}
\end{equation*}

der $\beta$ beskriver dempningskonstanten.  
Når $\beta = 0$ får vi den ideelle, tapsfrie bølgeligningen, mens $\beta > 0$ fører til at energien gradvis avtar.  
Demping påvirker hvor lenge svingninger varer, og spiller en viktig rolle i akustikk, bygg og elektronikk \parencite{libretextsDamping}.

Demping er utelatt fra den implementerte modellen fordi tidsrommet som blir simulert er så kort at energitapet er lite relevant.
En dempningsterm ville hovedsakelig påvirket hvor raskt amplituden avtar over tid, ikke selve bølgens form i starten av bevegelsen.
det blir likevel inkludert en omtale av fenomenet for å vise forståelse av hvordan virkelige systemer skiller seg fra den ideelle, tapsfrie bølgeligningen.

\subsection{Dispersjon}

Dispersjon beskriver fenomenet at bølger med ulike frekvenser beveger seg med forskjellige hastigheter i et medium.  
For ideelle strenger, som den klassiske bølgeligningen beskriver, er bølgefarten uavhengig av frekvensen, og det oppstår ingen dispersjon.  
I virkeligheten er imidlertid en gitarstreng ikke helt ideell – den har både stivhet og massefordeling som gjør at bølgefarten øker svakt med frekvensen \parencite{kartofelev2019dispersive}.  

Dette betyr at de høyeste overtonene på en gitarstreng forplanter seg litt raskere enn de lavere.  
Resultatet er at de høyeste frekvenskomponentene kommer ut av fase med de lavere, noe som fører til at lyden gradvis mister renheten sin.  
Fenomenet kalles \textit{inharmonisitet}, og det er en viktig årsak til at pianoer og gitarer må justeres (intoneres) for at akkorder skal klinge riktig.  

Den fysiske årsaken ligger i strengens stivhet, som legger til et ekstra ledd i bølgeligningen:  

\begin{equation*}
\frac{\partial^2 u}{\partial t^2} = c^2 \frac{\partial^2 u}{\partial x^2} - \kappa^2 \frac{\partial^4 u}{\partial x^4}
\end{equation*}

der $\kappa$ er en konstant som beskriver strengens stivhet.  
Dette leddet gjør at bølgefarten ikke lenger er konstant, men øker svakt med frekvensen.  
Dispersjon i gitarstrenger er derfor en praktisk påminnelse om hvordan virkelige materialer avviker fra den ideelle matematiske modellen — og hvordan dette påvirker instrumentets klang og stemning \parencite{kartofelev2019dispersive, unswStringsHarmonics, wikipediaInharmonicity}.
\clearpage
Dispersjon er ikke en del av modellen fordi det blir tatt utgangspunkt i den klassiske bølgeligningen, som antar en ideell, fleksibel streng uten stivhet.
Innenfor dette rammeverket er bølgefarten konstant og uavhengig av frekvens.
Det blir likevel nevt dispersjon for å understreke hvordan virkelige gitarstrenger avviker fra den ideelle modellen, og for å vise sammenhengen mellom matematisk teori og praktiske observasjoner av lyd og klang.

