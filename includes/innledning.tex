\subsection{Bakgrunn og målsetting}
Bølgeligningen er en av de mest fundamentale partielle differensialligningene i fysikk og matematikk. 
Den beskriver hvordan bølger forplanter seg i tid og rom, og opptrer i mange sammenhenger - fra 
vibrasjoner i en streng eller en membran, til akustiske bølger i luft, seismiske bølger i jordskorpen 
og elektromagnetiske bølger i vakuum. 

I dette prosjektet tar vi utgangspunkt i den klassiske éndimensjonale bølgeligningen:

\begin{equation*}
    \frac{\partial^2 u}{\partial t^2} = c^2 \frac{\partial^2 u}{\partial x^2}
    \label{eq:bølgeligningen}
\end{equation*}

der $u(x,t)$ beskriver utslaget til systemet, og $c$ er bølgefarten bestemt av materialparametere. 
Vi viser hvordan ligningen kan utledes fra enkle fysiske prinsipper, og hvordan ulike rand- og 
initialbetingelser gir opphav til stående bølger og egenfrekvenser. 

Videre undersøker vi hvordan løsningen oppfører seg under forskjellige forhold, blant annet hvordan 
resonans oppstår når en streng utsettes for en periodisk ytre kraft, og hvordan energien i systemet 
utvikler seg med eller uten demping. Resultatene vil bli både analytisk utledet og numerisk simulert, 
og vi vil sammenligne teori med numeriske observasjoner. 

Formålet med rapporten er dermed å belyse bølgeligningens matematiske struktur, demonstrere sentrale 
fysiske fenomener knyttet til bølger, og trekke paralleller til praktiske anvendelser. Rapporten er 
bygd opp slik at vi først presenterer den teoretiske bakgrunnen og utledningen, deretter beskriver vi 
numeriske metoder og simuleringer, før vi avslutter med resultater, diskusjon og konklusjon.

\subsection{Problemstilling og valg av tema}

I denne oppgaven har vi valgt å fokusere på gitarstrengen som et konkret eksempel på
bølgeligningen i praksis. Problemstillingen vi ønsker å undersøke er:

\begin{center}
    \textit{Hvordan kan strengens materiale, tykkelse og spennkraft modelleres i bølgeligningen, 
    og hvilken betydning har disse faktorene for bølgefarten og dermed tonehøyden på en gitar?}
\end{center}

Vi valgte denne problemstillingen fordi den kombinerer en fundamental matematisk modell 
med et fysisk og praktisk fenomen som de fleste kan relatere til. En gitarstreng er et lettfattelig
eksempel på hvordan bølger oppfører seg, og hvordan små endringer i fysiske parametere kan gi 
merkbare utslag i lyd og tonehøyde. Dermed får vi muligheten til både å fordype oss i 
den matematiske utledningen og samtidig knytte teorien til en konkret anvendelse i musikk.

Problemstillingen er interessant fordi den viser hvordan en enkel partielle differensialligning 
kan forklare komplekse og observerbare fenomener. Samtidig åpner den for en kombinasjon 
av analytiske utledninger og numeriske simuleringer, noe som gir oss mulighet til å sammenligne 
teori og praksis på en strukturert måte.

\subsubsection*{Hvordan vi skal gå frem}

Vi starter med den klassiske bølgeligningen, som beskriver utslaget $u(x,t)$ til en streng i rotnormal retning:

\begin{equation*}
  \frac{\partial^2 u}{\partial t^2} = c^2 \frac{\partial^2 u}{\partial x^2},
  \label{eq:bolgeligningen}
\end{equation*}

og viser hvordan denne kan utledes fra Newtons 2.~lov anvendt på et lite strengsegment.
Videre kobler vi bølgefarten $c$ til strengens fysiske parametere gjennom

\begin{equation*}
  c = \sqrt{\frac{T}{\mu}}, \qquad \mu = \rho A,
\end{equation*}

der $T$ er spennkraften i strengen, $\rho$ er materialets tetthet og $A$ er strengens tverrsnittsareal.
Som en løsning på bølgeligningen kan vi anta en harmonisk bølge av formen

\begin{equation*}
  u(x,t) = A \sin(kx - \omega t),
\end{equation*}

der $A$ er amplituden, $k=\tfrac{2\pi}{\lambda}$ er bølgetallet og $\omega = 2\pi f$ er vinkelfrekvensen.  
Sammenhengen mellom $\omega$, $k$ og $c$ følger fra dispersjonsrelasjonen

\begin{equation*}
  \omega = c k.
\end{equation*}

For en streng med lengde $L$ og faste ender ($u(0,t)=u(L,t)=0$) finner vi at egenfrekvensene blir

\begin{equation*}
  f_n = \frac{n}{2L}\,c, \qquad n=1,2,3,\dots
\end{equation*}

Dermed kan vi analysere hvordan variasjon i $T$, $\rho$ og $A$ påvirker både grunnfrekvensen og de høyere overtonene.
En vilkårlig strengutslag $u(x,0)$ kan videre uttrykkes som en sum av harmoniske svingninger ved hjelp av Fourier-rekker.
Den generelle løsningen for $u(x,t)$ kan uttrykkes med moduser 
$\sin\!\left(\tfrac{n\pi x}{L}\right)$, der tidsdynamikken for hver modus har formen 

\begin{equation*}
    u(x,t) = \sum_{n=1}^{\infty} \bigl[a_n \cos(\omega_n t) + b_n \sin(\omega_n t)\bigr] 
    \sin\!\left(\tfrac{n\pi x}{L}\right).
\end{equation*}

der koeffisientene $a_n$ og $b_n$ bestemmes av de gitte initialbetingelsene.
For en detaljert fremstilling og utledning av denne metoden, se LibreTexts: 
\textit{Solving the Wave Equation with Fourier Transforms} \parencite{libretextsWave}



Til slutt vil vi sammenligne de utledede teoretiske uttrykkene med numeriske simuleringer 
av bølgeligningen for representative parametere, for å bekrefte modellens gyldighet og illustrere praktiske effekter.

\subsection{Hvordan kunstig intelligens er brukt i arbeidet med rapporten}
I arbeidet med denne oppgaven har vi brukt kunstig intelligens som et supplement, hovedsakelig for å hjelpe med formatering, strukturering og oppsett i LaTeX. 
Selve de matematiske utledningene og formlene har vi gjort på egen hånd, ofte først på papir, før vi har satt dem inn i rapporten. 
Kunstig intelligens har derfor fungert som en støttespiller for å gjøre teksten mer oversiktlig og effektiv å presentere, men den faglige forståelsen, beregningene og drøftingene har vi selv stått for. 
På denne måten har vi sikret at oppgaven gjenspeiler vår egen innsats og kunnskap, samtidig som vi har brukt verktøyet til å styrke klarheten og kvaliteten på framstillingen.