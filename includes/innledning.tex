\subsection{Bakgrunn og målsetting}
Bølgeligningen er en av de mest fundamentale partielle differensialligningene i fysikk og matematikk. 
Den beskriver hvordan bølger forplanter seg i tid og rom, og opptrer i mange sammenhenger - fra 
vibrasjoner i en streng eller en membran, til akustiske bølger i luft, seismiske bølger i jordskorpen 
og elektromagnetiske bølger i vakuum. 

I dette prosjektet blir det tatt utgangspunkt i den klassiske éndimensjonale bølgeligningen:

\begin{equation}
    \frac{\partial^2 u}{\partial t^2} = c^2 \frac{\partial^2 u}{\partial x^2}
    \label{eq:bølgeligningen}
\end{equation}

der $u(x,t)$ beskriver utslaget til systemet, og $c$ er bølgefarten bestemt av materialparametere. 
Det blir vist hvordan ligningen kan utledes fra enkle fysiske prinsipper, og hvordan ulike rand- og 
initialbetingelser gir opphav til stående bølger og egenfrekvenser. 

Videre undersøkes det hvordan den éndimensjonale bølgeligningen kan brukes til å beskrive bevegelsen til 
en streng med faste ender. Gjennom analytiske utledninger blir det funnet løsninger som viser hvordan strengen 
svinger i bestemte mønstre, avhengig av dens fysiske parametere. Deretter implementerer vi en numerisk 
simulering for å sammenligne den teoretiske løsningen med den beregnede bevegelsen. 
På denne måten kan en vurdere hvor godt den analytiske modellen samsvarer med den numeriske 
representasjonen av strengens svingninger.

Formålet med rapporten er dermed å belyse bølgeligningens matematiske struktur, demonstrere sentrale 
fysiske fenomener knyttet til bølger, og trekke paralleller til praktiske anvendelser. Rapporten er 
bygd opp slik at vi først presenterer den teoretiske bakgrunnen og utledningen, deretter beskrives det 
numeriske metoder og simuleringer, før det avsluttes med resultater, diskusjon og konklusjon.

\subsection{Problemstilling og valg av tema}

I denne oppgaven er det blitt valgt å fokusere på gitarstrengen som et konkret eksempel på
bølgeligningen i praksis. Problemstillingen som undersøkes i oppgaven er:

\begin{center}
    \textit{Hvordan kan bølgeligningen brukes til å beskrive bevegelsen til en streng med faste ender, 
    og hvordan stemmer den analytiske løsningen overens med en numerisk simulering av strengens 
    svingninger?}
\end{center}

Denne problemstillingen ble valgt fordi den kombinerer en fundamental matematisk modell 
med et fysisk og praktisk fenomen som de fleste kan relatere til. En gitarstreng er et enkelt
eksempel på hvordan bølger oppfører seg, og hvordan små endringer i fysiske parametere kan gi 
merkbare utslag i lyd og tonehøyde. Dermed åpner muligheten til både å fordype seg i 
den matematiske utledningen og samtidig knytte teorien til en konkret anvendelse i musikk.

Problemstillingen er interessant fordi den viser hvordan en enkel partielle differensialligning 
kan forklare komplekse og observerbare fenomener. Samtidig åpner den for en kombinasjon 
av analytiske utledninger og numeriske simuleringer, noe som gir oss mulighet til å sammenligne 
teori og praksis på en strukturert måte.

\subsubsection*{Hvordan vi skal gå frem}

En starter med den klassiske bølgeligningen, som beskriver utslaget $u(x,t)$ til en streng i rotnormal retning:

\begin{equation*}
  \frac{\partial^2 u}{\partial t^2} = c^2 \frac{\partial^2 u}{\partial x^2}
  \label{eq:bolgeligningen}
\end{equation*}

og viser hvordan denne kan utledes fra Newtons 2.~lov anvendt på et lite strengsegment.
Videre kobler vi bølgefarten $c$ til strengens fysiske parametere gjennom

\begin{equation}
  c^2 = \frac{T}{\mu}, \qquad \mu \approx \rho \label{eq:bølgefart}
\end{equation}

der $T$ er spennkraften i strengen og $\rho$ er materialets tetthet. 
Tilnærmingen $\mu \approx \rho$ benyttes fordi strengen antas å ha 
jevn tetthet og liten variasjon i tverrsnitt, slik at massen per lengdeenhet kan betraktes som tilnærmet lik materialets egen tetthet.
Som en løsning på bølgeligningen kan en ta fourier rekke av formen

\begin{equation*}
    u(x,t) = \sum_{n=0}^{\infty} c_n 
    \sin\!\left(\tfrac{n\pi x}{L}\right)
    \cos\!\left(\tfrac{n\pi c}{L}t\right)
\end{equation*}

For en streng med lengde $L$ og faste ender ($u(0,t)=u(L,t)=0$) finner vi at egenfrekvensene blir

\begin{equation*}
  f_n = \frac{n}{2L}\,c, \qquad n=1,2,3,\dots \label{egenfrekvensene}
\end{equation*}
\clearpage

Dermed kan en analysere hvordan variasjon i $T$, $\rho$ og $A$ påvirker både grunnfrekvensen og de høyere overtonene.  
En vilkårlig strengutslag $u(x,0)$ kan videre uttrykkes som en sum av harmoniske svingninger ved hjelp av Fourier-rekker.  
Den generelle løsningen for $u(x,t)$ kan da uttrykkes som en superposisjon av moduser 
$\sin\!\left(\tfrac{n\pi x}{L}\right)$, der hver modus har en tidsavhengig del gitt ved 
$\cos\!\left(\tfrac{n\pi c}{L}t\right)$:

\begin{equation*}
    u(x,t) = \sum_{n=0}^{\infty} c_n 
    \sin\!\left(\tfrac{n\pi x}{L}\right)
    \cos\!\left(\tfrac{n\pi c}{L}t\right)
\end{equation*}

Koeffisientene $c_n$ bestemmes av de gitte initialbetingelsene.  
Fourier-rekken gjør det mulig å beskrive komplekse bølgeformer som en sum av enklere, harmoniske svingninger.  
Denne metoden gjør det mulig å uttrykke en strengs bevegelse som en superposisjon av mange enkeltsvingninger med ulike frekvenser og amplituder.  
Ved å bruke denne representasjonen kan man analysere hvordan ulike fysiske faktorer, som spennkraft og tetthet, påvirker bølgefarten og tonehøyden på strengen \parencite{libretextsWave}

Til slutt skal det sammenlignes av de utledede teoretiske uttrykkene med numeriske simuleringer 
av bølgeligningen for representative parametere, for å bekrefte modellens gyldighet og illustrere praktiske effekter.

\subsection{Hvordan kunstig intelligens er brukt i arbeidet med rapporten}
I arbeidet med denne oppgaven har det blitt brukt kunstig intelligens som et supplement, hovedsakelig som en sparrepartner og hjelpeverktøy.
Selve de matematiske utledningene og formlene er blitt gjort på egen hånd, ofte først på papir, før dette er satt inn i rapporten. 
Kunstig intelligens har derfor fungert som et støtteverktøy for å gjøre teksten mer oversiktlig og effektiv å presentere. Men den faglige forståelsen, beregningene og drøftingene 
er selv stått til ansvar for. På denne måten har det blitt sikret at oppgaven gjenspeiler egen innsats og kunnskap, samtidig som det er brukt 
verktøyet til å styrke klarheten og kvaliteten på framstillingen.