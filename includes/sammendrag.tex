Denne rapporten tar for seg den éndimensjonale bølgeligningen, en grunnleggende partielldifferensialligning som beskriver hvordan bølger forplanter seg i tid og rom. 
Målet med prosjektet har vært å bruke denne ligningen til å modellere bevegelsen til en streng med faste ender, og undersøke hvordan spennkraft, massetetthet og lengde påvirker bølgefarten og tonehøyden.

Utgangspunktet for arbeidet er en fysisk modell av en streng, hvor bølgeligningen utledes fra Newtons 2.~lov. 
Gjennom analytiske utledninger ved hjelp av separasjon av variable og Fourier-rekker ble det funnet løsninger som beskriver matematiske tilfeller. 
Disse teoretiske resultatene danner grunnlaget for forståelsen av hvordan en streng kan vibrere i bestemte ulike mønstre.

For å sammenligne teori og praksis ble bølgeligningen implementert numerisk i Python. 
Ved å simulere en plukkebevegelse av strengen ble det laget animasjoner som viser hvordan bølgene oppstår og reflekteres mellom endene. 
Resultatene viste god overensstemmelse mellom den analytiske løsningen og den numeriske simuleringen, og illustrerte tydelig hvordan variasjoner i spennkraft og tetthet påvirker bølgefarten.  

Prosjektet inkluderer også en teoretisk omtale av resonans, demping og dispersjon for å vise forståelse av hvordan virkelige systemer avviker fra den ideelle matematiske modellen. 
Samlet sett viser prosjektet hvordan enkle fysiske prinsipper kan beskrive komplekse bølgefenomener, og hvordan numeriske metoder kan brukes til å visualisere og forstå dynamiske prosesser i både fysikk og teknologi.
