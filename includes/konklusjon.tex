\subsection{Kort oppsummering av prosjektet}

\subsubsection{Kort sammendrag av prosjektets hensikt og mål}

Hensikten med prosjektet har vært å bruke den éndimensjonale bølgeligningen til å modellere en streng med faste ender, 
og å undersøke hvordan parametere som spennkraft, massetetthet og lengde påvirker bølgefarten og tonehøyden. 
Bølgeligningen ble utledet fra Newtons 2.~lov anvendt på et lite strengsegment, der sammenhengen mellom de fysiske størrelsene ble tydeliggjort gjennom forholdet 
$c^2 = \tfrac{T}{\rho}$. Dette gir et direkte uttrykk for hvordan bølgefarten avhenger av strengens spennkraft og massetetthet.  

Den analytiske løsningen ble funnet ved hjelp av separasjon av variable, og beskriver hvordan en bølge brer seg og reflekteres mellom faste endepunkter. 
Den numeriske delen av prosjektet besto i å simulere strengens bevegelse etter en plukkebevegelse, slik at vi kunne sammenligne teori og praksis. 
Resultatene viste god overensstemmelse mellom den analytiske modellen og den numeriske simuleringen, og illustrerte tydelig hvordan variasjoner i spennkraft og tetthet påvirker bølgefarten og utslaget.  

\subsubsection{Teoretisk utledning}
\begin{enumerate}
  \item Antakelser: små utslag, jevn spenning, konstant lineær massetetthet langs strengen. 
  \item Newtons 2.~lov på et differensielt strengsegment gir
  
  \begin{equation*}
    \frac{\partial^2 u}{\partial t^2} = \frac{T}{\rho}\,\frac{\partial^2 u}{\partial x^2} \equiv c^2 u_{xx},\quad c^2=\frac{T}{\rho}.
  \end{equation*}

  Dette identifiserer bølgefarten $c$ i form av spennkraft $T$ og lineær massetetthet $\rho$.
  \item Randbetingelser for faste ender: $u(0,t)=u(L,t)=0$ leder til egenverdiproblemet med egenfunksjoner $X_n(x)=\sin\!\left(\frac{n\pi x}{L}\right)$. 
  \item Separasjon av variable gir tidsløsninger med $\omega_n=\frac{n\pi c}{L}$ og generell løsning som Fourier-sum:
  
  \begin{equation*}
	u(x,t) = \sum_{n=0}^{\infty} c_n 
	\sin \left( \frac{n \pi}{l} x \right)
	\cos \left( \frac{n \pi c}{l} t \right)
  \end{equation*}

  der koeffisientene bestemmes av initialdata.
  \item Egenfrekvensene følger Mersennes relasjon $f_n=\frac{n}{2L}\sqrt{T/\mu}$ (med $\mu$ lineær massetetthet; i modellen betegnet som $\rho$). 
\end{enumerate}

\subsubsection{Numerisk implementasjon}

I den numeriske delen av prosjektet ble bølgeligningen implementert i Python ved bruk av blant annet \texttt{numpy} for beregninger og \texttt{matplotlib} for visualisering. 
Strengen ble delt opp i et gitt antall punkter langs lengden $L$, slik at forskyvningen $u(x,t)$ kunne beregnes i diskrete tidstrinn. 
Parametere som bølgefart $c$, spenning $T$, og massetetthet $\rho$ ble definert ut fra de teoretiske sammenhengene, slik at simuleringen kunne sammenlignes direkte med den analytiske løsningen. 

Som startbetingelse ble det valgt en trekantformet utslagsprofil som representerer en streng som plukkes på et bestemt punkt. Dette ga et realistisk bilde av hvordan bølgene 
oppstår og reflekteres mellom endene. Gjennom animasjonen ble det tydelig hvordan stående bølger dannes, og hvordan noder og buker opptrer akkurat der teorien forutsier det. 
Den numeriske løsningen viste dermed god overensstemmelse med den teoretiske modellen på et beskrivende nivå, selv om små avvik kunne observeres på grunn av den diskrete tids- og romoppløsningen.

\subsection{Hva har vi lært?}
\subsubsection{Hva vi har lært om differensialligninger og numeriske metoder}

Gjennom dette prosjektet har vi lært hvordan partielle differensialligninger kan brukes til å beskrive bølgefenomener i fysiske systemer. Den éndimensjonale bølgeligningen viser 
hvordan en liten forskyvning i et punkt på en streng påvirker nabopunktene, og hvordan disse bevegelsene forplanter seg som bølger. Ved å utlede ligningen fra Newtons 2.~lov ble det tydelig 
hvordan matematiske prinsipper henger sammen med fysiske størrelser som spenning, tetthet og akselerasjon. 

Vi har også fått erfaring med hvordan slike ligninger kan løses både analytisk og numerisk. Den analytiske løsningen, der man bruker separasjon av variable og Fourier-rekker, gir innsikt 
i hvordan systemet kan beskrives som en sum av stående bølger med bestemte egenfrekvenser. Samtidig viste den numeriske metoden hvordan man kan beregne og visualisere bevegelsen steg for steg i tid. 
Dette gjorde det lettere å forstå hvordan kontinuerlige prosesser kan beskrives ved hjelp av diskrete beregninger, og hvor viktig valg av tidssteg og romoppløsning er for stabiliteten og nøyaktigheten til løsningen.

\subsubsection{Hvordan implementere numeriske metoder kan øke forståelsen av teoretiske konsepter}

Ved å implementere bølgeligningen numerisk fikk vi en mer intuitiv forståelse av de teoretiske resultatene. Når vi så den simulerte strengen bevege seg, ble det lettere å knytte 
matematiske uttrykk som egenmoduser og noder til faktiske fysiske fenomener. Visualiseringene gjorde det også enklere å forstå hvordan forskjellige startbetingelser påvirker hvilke bølgeformer som oppstår, 
og hvordan energien fordeler seg mellom modene. 

Den numeriske implementasjonen fungerte som et bindeledd mellom teori og praksis. Der de teoretiske utledningene kan virke abstrakte, gir den numeriske løsningen et konkret bilde 
på hvordan systemet oppfører seg over tid. Det ble tydelig at små endringer i parametere, som økt spenning eller redusert massetetthet, fører til høyere bølgefart og dermed høyere tonehøyde — akkurat slik teorien forutsier. 
På denne måten har vi erfart at numeriske metoder ikke bare er et verktøy for å finne løsninger, men også et middel for å styrke forståelsen av de fysiske prinsippene som ligger bak.

\subsection{Har målene blitt nådd?}

Prosjektet har i hovedsak nådd målene som ble satt i innledningen. 
Det har blitt vist hvordan den éndimensjonale bølgeligningen kan brukes til å beskrive bevegelsen til en streng med faste ender, 
og hvordan parametere som spennkraft, massetetthet og lengde påvirker bølgefarten og tonehøyden. 
Den teoretiske utledningen gav et klart uttrykk for sammenhengen mellom de fysiske størrelsene, 
og den numeriske simuleringen visualiserte hvordan bølgene oppstår og reflekteres mellom endepunktene.  

Resultatene fra simuleringen stemte godt overens med den analytiske løsningen og bekreftet den forventede avhengigheten:

\begin{equation*}
  c^2 = \frac{T}{\rho}
\end{equation*}
Selv om sammenligningen mellom teori og numerikk kunne vært gjort mer kvantitativt, 
har prosjektet gitt en helhetlig forståelse av hvordan bølgeligningen beskriver strengens svingninger, 
og hvordan matematiske modeller kan brukes til å forklare observerbare bølgefenomener.


