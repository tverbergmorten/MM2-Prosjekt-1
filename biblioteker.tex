% --- Fjerne unødvendige warnings ---
\usepackage{silence}

% --- Font- og inndatakoding ---
\usepackage[T1]{fontenc}                                % Europeisk fontkoding (korrekte æ/ø/å i PDF)
\usepackage[utf8]{inputenc}                             % Inndata-koding (utf8) for pdflatex


% --- Matematikk (unngå redef-kollisjoner) ---
\usepackage{amsmath}                                    % Bedre matemodus, align/multline/gather
\usepackage{amssymb}                                    % Ekstra matematiske symboler
\usepackage{amsthm}                                     % Teorem-, lemma- og bevis-miljøer
\usepackage{mathtools}                                  % Utvidelser til amsmath (f.eks. \coloneqq)
\usepackage{bm}                                         % Fete greske bokstaver (\bm)

% --- Skrifter, mikrotypografi, bilder ---
\usepackage{newtxtext}                                  % Times-lignende brødtekst
\usepackage[nopatch=footnote]{microtype}                % Mikrosperring/kerning for penere avsnitt
\usepackage{graphicx}                                   % Inkludere bilder (\includegraphics)

% --- Struktur og lister ---
\usepackage{titlesec}                                   % Tilpasse seksjonsoverskrifter
\usepackage{enumitem}                                   % Kontroll på listeinnrykk/spacing

% --- Lenker og referanser ---
\usepackage[unicode]{hyperref}                          % Klikkbare lenker/refs i PDF, metadata

% --- Sideoppsett og enheter ---
\usepackage[a4paper,margin=2.5cm]{geometry}             % Sidestørrelse og marger
\usepackage{siunitx}                                    % SI-enheter og tallformatering (\SI, \num)
\usepackage{setspace}                                   % Justere linjeavstand (\setstretch)

% --- Innholdsfortegnelse og fysikknotasjon ---
\usepackage{tocloft}                                    % Finjustere innholdsfortegnelsen
\usepackage{physics}                                    % Vanlige fysikkmakroer (\dv, \qty, …)

% --- Litteratur ---
\usepackage{csquotes}                                   % Korrekte sitater (anbefalt for biblatex)%
\usepackage[backend=biber,style=ieee]{biblatex}         % Kildehåndtering m/IEEE-stil



% --- Språk ---
\usepackage[main=norsk,english]{babel}              % Orddeling/språk (norsk hovedspråk)

% --- Grafer ---
\usepackage{tikz}                                       % Tegne grafer / piler osv.
\usepackage{pgfplots}                                   % Plotte grafer 
\usetikzlibrary{arrows.meta,calc,angles,quotes}
\pgfplotsset{compat=1.18}
\usepackage{tkz-euclide}

% --- Kode ---
\usepackage{listings}                                   % Inkludere kode med syntax highlighting
\usepackage{xcolor}

% --- Objektorientering i dokumentet ---
\usepackage{float}                                      % Bedre kontroll over plassering av figurer/tabeller